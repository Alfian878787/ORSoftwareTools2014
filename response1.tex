\documentclass[12pt, letterpaper]{article}
\usepackage[top=0.75in, bottom=0.75in, left=0.75in, right=0.75in]{geometry}

\usepackage{multirow}
\usepackage{graphicx}
\usepackage{caption}
\usepackage{subcaption}
\usepackage{epstopdf}
\usepackage{footnote}
\usepackage{fmtcount}
\usepackage{array}
\usepackage{setspace}
\usepackage{amsmath}

\usepackage{natbib}
 \bibpunct[, ]{(}{)}{,}{a}{}{,}%
 \def\bibfont{\small}%
 \def\bibsep{\smallskipamount}%
 \def\bibhang{24pt}%
 \def\newblock{\ }%
 \def\BIBand{and}%
 
\title{Response to Reviewer Comments for ITE-2014-05-0006.R1}
\author{``Advanced Software Tools for Operations Research and Analytics''}
\date{}
\begin{document}
\maketitle

We thank the reviewer for the feedback on our submission. Our revised submission addresses all reviewer comments, as detailed below:

\vspace{0.5cm}

\noindent\textit{1. Covering each of these modules in a single three-hour class seems to be exceptionally ambitious. I suggest the authors spend more time describing the expected backgrounds for students in a class like this. What is the minimum knowledge base and experience that students must possess to make this approach feasible? How do the instructors of this class handle the variability in student abilities in the class? Are teams assigned to balance abilities? Are students ``brought up to speed" in some manner? This should all be more thoroughly addressed.}

We updated Section 1 to describe the target audience, which is graduate students who have already taken coursework in machine learning and optimization and who have some programming background (though not necessarily in any of the tools taught in the course). We updated Section 2.1 to discuss how in-class collaboration can support students with weaker programming skills and how the bonus exercises that accompanied nearly every in-class exercise kept stronger programmers busy. Finally, we updated Section 4 to discuss how the course could be modified to apply to students with a weaker background either in the course topics or in programming. 

\vspace{0.5cm}

\noindent\textit{2. Related to the point above, what pre-work is assigned prior to each module?}

We added a paragraph to Section 4 that describes the pre-work assigned prior to each module and the rationale for this pre-work.

\vspace{0.5cm}

\noindent\textit{3. Also related to the above, which of these modules are then covered in greater detail in full courses? Are full courses offered in Data Wrangling, Data Visualization, etc. If so, some discussion of how the offering of this survey course has impacted student experiences in those other courses may be appropriate.}

\textbf{TODO: Response}

\vspace{0.5cm}

\noindent\textit{4. I think it would be helpful to include a ``Lessons Learned" section in the paper. The authors do discuss some changes made from the first to the second offering of the course, but I think a more general summary of what they would change going forward in the course (even after two offerings) would be helpful.}

\textbf{TODO: Response}

\vspace{0.5cm}

\noindent\textit{5. The authors should spend a little time discussing assessment. How is grading done for the course? Are all assignments team-based? How are individuals graded?}

We added a paragraph to Section 4 describing the assessments we used and additional opportunities to assess students.

\vspace{0.5cm}

\noindent\textit{6. The authors cite feedback from a student on page 7: ``I like \ldots that the class \ldots covered many common tools without dwelling on explaining them." I wonder what evidence the authors have that this is an effective form of learning for students. Is this a reference to ``active learning?" Or something else?}

We have clarified the wording in this paragraph; we found that avoiding detailed descriptions of statistical methods was effective because most students in the class were already comfortable with the methods.

\vspace{0.5cm}

\noindent\textit{7. The paper discusses tools for ``operations research," but I actually would favor a change to discussing this in the broader context of ``analytics." Several of the tasks discussed here (including data wrangling, data visualization, etc.) have not been traditionally covered as part of OR curricula, but are definitely part of analytics. Given INFORMS rebranding as ``analytics" and
the broad coverage of this course, I would replace most uses of ``operations research" with
``analytics," but this is only a suggestion.}

We thank the reviewer for this helpful suggestion. As stated in this comment, some of the aspects of this work such as data wrangling and visualization are more closely aligned with analytics than traditional OR. At the same time, we believe that some of the aspects of this course such as Advanced MILP techniques are more closely aligned with traditional OR than with analytics. As a result, we have adjusted the terminology throughout the paper to refer to the topic of the course as ``operations research and analytics."

\vspace{0.5cm}

\noindent\textit{8a. Page 2, 1st Full Paragraph: The authors should cite examples of ``many computer science programs offer courses on visualization." I (and I believe other readers) would be interested in examples of such courses.}

\textbf{TODO: Response}

\vspace{0.5cm}

\noindent\textit{8b. Page 8, Line 5: should read ``visualization and \textbf{we} used examples\ldots"}

We have made this update.

\vspace{0.5cm}

\noindent\textit{8c. In Figure 1, it is not clear what is meant by the directed arrows in the figure. Does this mean that, for instance, Visualization reinforces Machine Learning in R, but not vice versa? This should be explained.}

We have clarified the meaning on the directed edges in Figure 1 both in the figure caption and in the text of the referencing section.

\end{document}
